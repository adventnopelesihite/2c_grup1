\section{Bakti Qilan Mufid | 1174083}
\subsection{Buatlah library fungsi (file terpisah/library dengan nama NPM\textunderscore bar.py) untuk plot dengan jumlah subplot adalah NPM mod 3 + 2}
\hfill \break
Pertama-tama kita harus mengetahui hasil dari NPM mod 3 + 2 terlebih dahulu, lalu sesudah itu kita membuat subplotnya, seperti pada kode berikut:
\lstinputlisting[firstline=9, lastline=79]{src/6/1174083/Praktek/1174083_bar.py}

\subsection{Buatlah library fungsi (file terpisah/library dengan nama NPM\textunderscore scatter.py) untuk plot dengan jumlah subplot adalah NPM mod 3 + 2}
\hfill \break
Pertama-tama kita harus mengetahui hasil dari NPM mod 3 + 2 terlebih dahulu, lalu sesudah itu kita membuat subplotnya, seperti pada kode berikut:
\lstinputlisting[firstline=9, lastline=79]{src/6/1174083/Praktek/1174083_scatter.py}

\subsection{Buatlah library fungsi (file terpisah/library dengan nama NPM\textunderscore pie.py) untuk plot dengan jumlah subplot adalah NPM mod 3 + 2}
\hfill \break
Pertama-tama kita harus mengetahui hasil dari NPM mod 3 + 2 terlebih dahulu, lalu sesudah itu kita membuat subplotnya, seperti pada kode berikut:
\lstinputlisting[firstline=9, lastline=54]{src/6/1174083/Praktek/1174083_pie.py}

\subsection{Buatlah library fungsi (file terpisah/library dengan nama NPM\textunderscore plot.py) untuk plot dengan jumlah subplot adalah NPM mod 3 + 2}
\hfill \break
Pertama-tama kita harus mengetahui hasil dari NPM mod 3 + 2 terlebih dahulu, lalu sesudah itu kita membuat subplotnya, seperti pada kode berikut:
\lstinputlisting[firstline=9, lastline=79]{src/6/1174083/Praktek/1174083_plot.py}

\subsection{Keterampilan Penanganan Error}
Fungsi Penanganan error sebagai berikut:
\lstinputlisting[firstline=9, lastline=26]{src/6/1174083/Praktek/error.py}

%%%%%%%%%%%%%%%%%%%%%%%%%%%%%%%%%%%%%%%%%%%%%%%%%%%%%%%%%%%%%%%%%%%%%%%%%%%%%%%%%%%%%%%%%%%%%%%%%%%%%%%%%%%%%%%%